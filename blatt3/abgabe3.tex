% LaTeX Template für Abgaben an der Universität Stuttgart
% Autor: Sandro Speth
% Bei Fragen: Sandro.Speth@studi.informatik.uni-stuttgart.de
%-----------------------------------------------------------
% Hauptmodul des Templates: Hier können andere Dateien eingebunden werden
% oder Inhalte direkt rein geschrieben werden.
% Kompiliere dieses Modul um eine PDF zu erzeugen.

% Dokumentenart. Ersetze 12pt, falls die Schriftgröße anzupassen ist.
\documentclass[12pt]{scrartcl}
% Einbinden der Pakete, des Headers und der Formatierung.
% Mit den \include und \input Befehlen können Dateien eingebunden werden:
% \include: Fügt einen Seitenumbruch nach dem Text ein
% \input: Fügt KEINEN Seitenumbruch nach dem Text ein
\input{../styles/Packages.tex}
\input{../styles/FormatAndHeader.tex}

% Counter für das Blatt und die Aufgabennummer.
% Ersetze die Nummer des Übungsblattes und die Nummer der Aufgabe
% den Anforderungen entsprechend.
% Definiert werden die Counter in FormatAndHeader.tex
% Beachte:
% \setcounter{countername}{number}: Legt den Wert des Counters fest
% \stepcounter{countername}: Erhöht den Wert des Counters um 1.
\setcounter{sheetnr}{3} % Nummer des Übungsblattes
\setcounter{exnum}{1} % Nummer der Aufgabe

\usetikzlibrary{chains,positioning}

% Beginn des eigentlichen Dokuments
\begin{document}

% Nutze den \exercise{Aufgabenname} Befehl, um eine neue Aufgabe zu beginnen.
% Möchtest du eine Aufgabe in der Nummerierung überspringen, schreibe vor der Aufgabe: \stepcounter{exnum}
% Möchtest du die Nummer einer Aufgabe auf eine beliebige Zahl x setzen, schreibe vor der Aufgabe: \setcounter{exnum}{x}
\setcounter{exnum}{1}
\exercise{Perlman-Algorithmus}
    \begin{enumerate}[label=(\alph*)]
        \item   Der Spannbaum dient der Vermeidung von Zyklen beim Fluten von Nachrichten innerhalb eines Direktverbindungsnetzes.
        \item  \ \\ \includegraphics[width = 0.9 \textwidth]{grafiken/1b.png}
        \item  \ \\  \includegraphics[width = 0.9 \textwidth]{grafiken/1c.png}
        \newpage
        \item  \ \\ \includegraphics[width = 0.9 \textwidth]{grafiken/1d.png}
    \end{enumerate}
 
\setcounter{exnum}{2}
\exercise{Distanz-Vektor-Routing}
    \newcommand{\un}{$\infty$}
    \textit{Pfadangabe wurde bei direktem Weg der übersichtshalber ausgelassen}
    \begin{table}[h]
        \centering
        \begin{tabularx}{\textwidth}{|p{2.75cm}|X|X|X|X|X|X|X|}\hline
            & \multicolumn{7}{c|}{Entfernung zu ... (Über)} \\\hline
            Information gespeichert auf Knoten... 
                & A     & B     & C     & D     & E     & F     & G     \\\hline
            A   & 0     & 5     & 1     & 9     & \un   & \un   & \un   \\\hline 
            B   & 5     & 0     & \un   & 4     & \un   & \un   & \un   \\\hline
            C   & 1     & \un   & 0     & \un   & 2     & \un   & \un   \\\hline
            D   & 9     & 4     & \un   & 0     & \un   & 8     & \un   \\\hline
            E   & \un   & \un   & 2     & \un   & 0     & 7     & 2     \\\hline
            F   & \un   & \un   & \un   & 8     & 7     & 0     & 1     \\\hline
            G   & \un   & \un   & \un   & \un   & 2     & 1     & 0     \\\hline
        \end{tabularx}
        \caption{Routingtabelle nach der ersten Iteration (Iteration 0)}
    \end{table}
    \begin{table}[h]
        \centering
        \begin{tabularx}{\textwidth}{|p{2.75cm}|X|X|X|X|X|X|X|}\hline
            & \multicolumn{7}{c|}{Entfernung zu... (Über)} \\\hline
            Information gespeichert auf Knoten... 
                & A     & B     & C     & D     & E     & F     & G     \\\hline
            A   & 0     & 5     & 1     & 9 (B) & 3 (C) & 17 (D)& \un   \\\hline
            B   & 5     & 0     & 6 (A) & 4     & \un   & 12 (D)& \un   \\\hline
            C   & 1     & 6 (A) & 0     & 10 (A)& 2     & 9 (E) & 4 (E) \\\hline
            D   & 9     & 4     & 10 (A)& 0     & 15 (F)& 8     & 9 (F) \\\hline
            E   & 3 (C) & \un   & 2     & 15 (F)& 0     & 3 (G) & 2     \\\hline
            F   & 17 (D)& 12 (D)& 9 (E) & 8     & 3 (G) & 0     & 1     \\\hline
            G   & \un   & \un   & 4 (E) & 9 (F) & 2     & 1     & 0     \\\hline 
        \end{tabularx}
        \caption{Routingtabelle nach der zweiten Iteration (Iteration 1)}
    \end{table}
    \begin{table}[h]
        \centering
        \begin{tabularx}{\textwidth}{|p{2.75cm}|X|X|X|X|X|X|X|}\hline
            & \multicolumn{7}{c|}{Entfernung zu... (Über)} \\\hline
            Information gespeichert auf Knoten... 
                & A     & B     & C     & D     & E     & F     & G     \\\hline
            A   & 0     & 5     & 1     & 9 (B) & 3 (C) & 10 (C)& 5 (C) \\\hline
            B   & 5     & 0     & 6(A)  & 4     & 8 (A) & 12 (D)& 13 (D)\\\hline
            C   & 1     & 6 (A) & 0     & 10 (A)& 2     & 5 (E) & 4 (E) \\\hline
            D   & 9     & 4     & 10 (A)& 0     & 11 (F)& 8     & 9 (F) \\\hline
            E   & 3 (C) & 8 (C) & 2     & 11 (G)& 0     & 3 (G) & 2     \\\hline
            F   & 10 (E)& 12 (D)& 5(G)  & 8     & 3 (G) & 0     & 1     \\\hline
            G   & 5 (E) & 13 (F)& 4(E)  & 9 (F) & 2     & 1     & 0     \\\hline
        \end{tabularx}
        \caption{Routingtabelle nach der dritten Iteration (Iteration 2)}
    \end{table}

% Ende des Dokuments
\end{document}