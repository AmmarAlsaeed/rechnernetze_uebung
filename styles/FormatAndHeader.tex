% LaTeX Template für Abgaben an der Universität Stuttgart
% Autor: Sandro Speth
% Bei Fragen: Sandro.Speth@studi.informatik.uni-stuttgart.de
%-----------------------------------------------------------
% Modul beinhaltet Befehl fuer Aufgabennummerierung,
% sowie die Header Informationen.

% Überschreibt enumerate Befehl, sodass 1. Ebene Items mit
\renewcommand{\theenumi}{\arabic{enumi}.}
% (a), (b), etc. nummeriert werden.
\renewcommand{\labelenumi}{\text{\theenumi}}

% Counter für das Blatt und die Aufgabennummer.
% Ersetze die Nummer des Übungsblattes und die Nummer der Aufgabe
% den Anforderungen entsprechend.
% Gesetz werden die counter in der hauptdatei, damit siese hier nicht jedes mal verändert werden muss
% Beachte:
% \setcounter{countername}{number}: Legt den Wert des Counters fest
% \stepcounter{countername}: Erhöht den Wert des Counters um 1.
\newcounter{sheetnr}
\newcounter{exnum}

% Befehl für die Aufgabentitel
\newcommand{\exercise}[1]{\section*{Aufgabe \theexnum\stepcounter{exnum}: #1}} % Befehl für Aufgabentitel

% Formatierung der Kopfzeile
% \ohead: Setzt rechten Teil der Kopfzeile mit
% Namen und Matrikelnummern aller Bearbeiter
\ohead{Alexander Kharitonov (3390885)\\
       Thomas Oßwald (3307917)\\
       Angela Kächele (3260892)\\
       Ammar Alsaeed (3408120)}

% \chead{} kann mittleren Kopfzeilen Teil sezten
% \ihead: Setzt linken Teil der Kopfzeile mit
% Modulnamen, Semester und Übungsblattnummer
\ihead{Rechnernetze I\\
Sommersemester 2021\\
Übungsblatt \thesheetnr}
