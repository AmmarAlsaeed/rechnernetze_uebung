% LaTeX Template für Abgaben an der Universität Stuttgart
% Autor: Sandro Speth
% Bei Fragen: Sandro.Speth@studi.informatik.uni-stuttgart.de
%-----------------------------------------------------------
% Hauptmodul des Templates: Hier können andere Dateien eingebunden werden
% oder Inhalte direkt rein geschrieben werden.
% Kompiliere dieses Modul um eine PDF zu erzeugen.

% Dokumentenart. Ersetze 12pt, falls die Schriftgröße anzupassen ist.
\documentclass[12pt]{scrartcl}
% Einbinden der Pakete, des Headers und der Formatierung.
% Mit den \include und \input Befehlen können Dateien eingebunden werden:
% \include: Fügt einen Seitenumbruch nach dem Text ein
% \input: Fügt KEINEN Seitenumbruch nach dem Text ein
% LaTeX Template für Abgaben an der Universität Stuttgart
% Autor: Sandro Speth
% Bei Fragen: Sandro.Speth@studi.informatik.uni-stuttgart.de
%-----------------------------------------------------------
% Modul fuer verwendete Pakete.
% Neue Pakete einfach einfuegen mit dem \usepackage Befehl:
% \usepackage[options]{packagename}

\usepackage{ifxetex}
\usepackage{ifluatex}
% setting up a helper macro
\newif\ifxetexorluatex
\ifxetex
    \xetexorluatextrue
\else
    \ifluatex
        \xetexorluatextrue
    \else
        \xetexorluatexfalse
    \fi
\fi
% using appropiate locale settings for latex compiler
\ifxetexorluatex
    \usepackage{polyglossia}
    \usepackage{csquotes}
    \setdefaultlanguage[babelshorthands=true, spelling=new]{german}
    \setotherlanguage{english}
\else
    \usepackage{ngerman}[babel]
    \usepackage[T1]{fontenc}
    \usepackage[utf8]{inputenc}
    \usepackage{dsfont}
    \usepackage{lmodern}
\fi

\usepackage{graphicx}
\usepackage[hyperref,dvipsnames]{xcolor}
\usepackage{listings}
\usepackage[a4paper,lmargin={2cm},rmargin={2cm},tmargin={3.5cm},bmargin = {2.5cm},headheight = {4cm}]{geometry}
\usepackage{amsmath,amssymb,amstext,amsthm}
\usepackage{stmaryrd}
\usepackage[lined,algonl,boxed]{algorithm2e}
% alternative zu algorithm2e:
%\usepackage[]{algorithm} %counter mit chapter
%\usepackage{algpseudocode}
\usepackage{tikz}
\usepackage{hyperref}
\usepackage{url}
\usepackage[inline]{enumitem} % Ermöglicht ändern der enum Item Zahlen
\usepackage[headsepline]{scrlayer-scrpage}
\pagestyle{scrheadings}
\usetikzlibrary{automata,positioning}
\usepackage[binary-units]{siunitx}
\usepackage{tabularx}
% LaTeX Template für Abgaben an der Universität Stuttgart
% Autor: Sandro Speth
% Bei Fragen: Sandro.Speth@studi.informatik.uni-stuttgart.de
%-----------------------------------------------------------
% Modul beinhaltet Befehl fuer Aufgabennummerierung,
% sowie die Header Informationen.

% Überschreibt enumerate Befehl, sodass 1. Ebene Items mit
\renewcommand{\theenumi}{\arabic{enumi}.}
% (a), (b), etc. nummeriert werden.
\renewcommand{\labelenumi}{\text{\theenumi}}

% Counter für das Blatt und die Aufgabennummer.
% Ersetze die Nummer des Übungsblattes und die Nummer der Aufgabe
% den Anforderungen entsprechend.
% Gesetz werden die counter in der hauptdatei, damit siese hier nicht jedes mal verändert werden muss
% Beachte:
% \setcounter{countername}{number}: Legt den Wert des Counters fest
% \stepcounter{countername}: Erhöht den Wert des Counters um 1.
\newcounter{sheetnr}
\newcounter{exnum}

% Befehl für die Aufgabentitel
\newcommand{\exercise}[1]{\section*{Aufgabe \theexnum\stepcounter{exnum}: #1}} % Befehl für Aufgabentitel

% Formatierung der Kopfzeile
% \ohead: Setzt rechten Teil der Kopfzeile mit
% Namen und Matrikelnummern aller Bearbeiter
\ohead{Alexander Kharitonov (3390885)\\
       Thomas Oßwald (3307917)\\
       Angela Kächele (3260892)}

% \chead{} kann mittleren Kopfzeilen Teil sezten
% \ihead: Setzt linken Teil der Kopfzeile mit
% Modulnamen, Semester und Übungsblattnummer
\ihead{Rechnernetze I\\
Sommersemester 2021\\
Übungsblatt \thesheetnr}

% Counter für das Blatt und die Aufgabennummer.
% Ersetze die Nummer des Übungsblattes und die Nummer der Aufgabe
% den Anforderungen entsprechend.
% Definiert werden die Counter in FormatAndHeader.tex
% Beachte:
% \setcounter{countername}{number}: Legt den Wert des Counters fest
% \stepcounter{countername}: Erhöht den Wert des Counters um 1.
\setcounter{sheetnr}{2} % Nummer des Übungsblattes
\setcounter{exnum}{1} % Nummer der Aufgabe

% Beginn des eigentlichen Dokuments
\begin{document}

% Nutze den \exercise{Aufgabenname} Befehl, um eine neue Aufgabe zu beginnen.
% Möchtest du eine Aufgabe in der Nummerierung überspringen, schreibe vor der Aufgabe: \stepcounter{exnum}
% Möchtest du die Nummer einer Aufgabe auf eine beliebige Zahl x setzen, schreibe vor der Aufgabe: \setcounter{exnum}{x}
\setcounter{exnum}{1}
\exercise{Hamming-Codes \& Rahmenbildung}
    \begin{enumerate}[label=(\alph*)]
        \item   Hamming-Abstand von A zu B: 5\\
                Hamming-Abstand von B zu C: 3\\
                Hamming-Abstand von A zu C: 6\\
                Der Hamming-Abstand von h ist also 3.
        \item   Mit D = 10100010 lässt sich der maximale Hamming-Abstand von 4 erzielen.
        \item   Mit einem Hamming-Abstand von 3 lassen sich Fehler bis zur Länge 2 erkennen (mit d > F).
        \item   Mit einem Hamming-Abstand von 3 lässt sich Fehler bis zur Länge 1 korrigieren (mit d > 2F)\\
    \end{enumerate}
 
\setcounter{exnum}{2}
\exercise{Fehlererkennung durch CRC}
    \begin{enumerate}[label=(\alph*)]
        \item Polynomdivision mit $F(x) \bmod G_{USB}(x)$:\\
        \begin{tabular}{c c c c c c c c c c c}
        1 & 0 & 0 & 0 & 0 & 0 & 1 & 0 & 0 & 0 & 1 \\
        1 & 0 & 0 & 1 & 0 & 1 \\
        \hline
          &   &   & 1 & 0 & 1 & 1 & 0 & 0 \\
          &   &   & 1 & 0 & 0 & 1 & 0 & 1 \\
        \hline
          &   &   &   &   & 1 & 0 & 0 & 1 & 0 & 1 \\
          &   &   &   &   & 1 & 0 & 0 & 1 & 0 & 1 \\
        \hline
          &   &   &   &   &   &   &   &   &   & 0 \\
        \end{tabular}\\
        Dh. der Fehler kann nicht erkannt werden, da $F(x) \bmod G_{USB}(x) = 0$.
        \item   Da die Länge des Bündelfehlers k größer ist als der Grad des Generatorpolynoms r, 
        kann der Fehler nicht sicher erkannt werden. \\
    \end{enumerate}

\setcounter{exnum}{3}
\exercise{Flusskontrolle}
    \begin{enumerate}[label=(\alph*)]
        \item   Zunächst sollte die Frame transmission time berechnet werden (Annahme: 1MB = 1000kB): 
        \begin{align*}
            T_{it} &= \frac{20 MB}{100 kB/s} = 200 s
        \end{align*}\\
        Damit lässt sich die Kanalauslastung berechnen (Annahme: $T_{ic}, \, T_{ac}\, $und $ T_{at}$ vernachlässigbar):\\
        \begin{align*}
            U &= \frac{T_{it}}{T_{ip} + T_{it} + T_{ap}}\\
            &= \frac{200 s}{300 s + 200 s + 300s}\\
            &= \frac{200s}{800s} = \frac{1}{4}
        \end{align*}
        \item   \begin{enumerate}[label=(\roman*)]
            \item bruh thomas trägt ein
            \item Für Go-Back-N mit $N\rightarrow \infty$ senden wir $N_{Go-Back-N} = \frac{7}{4} N$ Rahmen.\\
                Für Selective-Repeat mit $N\rightarrow \infty$ senden wir $N_{Sel.-Repeat} = \frac{3}{2} N$ Rahmen.\\
                Also werden für Go-Back-N $\frac{1}{4} N$ mehr Rahmen gesendet.
        \end{enumerate}
    \end{enumerate}

% Ende des Dokuments
\end{document}