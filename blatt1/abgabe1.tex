% LaTeX Template für Abgaben an der Universität Stuttgart
% Autor: Sandro Speth
% Bei Fragen: Sandro.Speth@studi.informatik.uni-stuttgart.de
%-----------------------------------------------------------
% Hauptmodul des Templates: Hier können andere Dateien eingebunden werden
% oder Inhalte direkt rein geschrieben werden.
% Kompiliere dieses Modul um eine PDF zu erzeugen.

% Dokumentenart. Ersetze 12pt, falls die Schriftgröße anzupassen ist.
\documentclass[12pt]{scrartcl}
% Einbinden der Pakete, des Headers und der Formatierung.
% Mit den \include und \input Befehlen können Dateien eingebunden werden:
% \include: Fügt einen Seitenumbruch nach dem Text ein
% \input: Fügt KEINEN Seitenumbruch nach dem Text ein
% LaTeX Template für Abgaben an der Universität Stuttgart
% Autor: Sandro Speth
% Bei Fragen: Sandro.Speth@studi.informatik.uni-stuttgart.de
%-----------------------------------------------------------
% Modul fuer verwendete Pakete.
% Neue Pakete einfach einfuegen mit dem \usepackage Befehl:
% \usepackage[options]{packagename}

\usepackage{ifxetex}
\usepackage{ifluatex}
% setting up a helper macro
\newif\ifxetexorluatex
\ifxetex
    \xetexorluatextrue
\else
    \ifluatex
        \xetexorluatextrue
    \else
        \xetexorluatexfalse
    \fi
\fi
% using appropiate locale settings for latex compiler
\ifxetexorluatex
    \usepackage{polyglossia}
    \usepackage{csquotes}
    \setdefaultlanguage[babelshorthands=true, spelling=new]{german}
    \setotherlanguage{english}
\else
    \usepackage{ngerman}[babel]
    \usepackage[T1]{fontenc}
    \usepackage[utf8]{inputenc}
    \usepackage{dsfont}
    \usepackage{lmodern}
\fi

\usepackage{graphicx}
\usepackage[hyperref,dvipsnames]{xcolor}
\usepackage{listings}
\usepackage[a4paper,lmargin={2cm},rmargin={2cm},tmargin={3.5cm},bmargin = {2.5cm},headheight = {4cm}]{geometry}
\usepackage{amsmath,amssymb,amstext,amsthm}
\usepackage{stmaryrd}
\usepackage[lined,algonl,boxed]{algorithm2e}
% alternative zu algorithm2e:
%\usepackage[]{algorithm} %counter mit chapter
%\usepackage{algpseudocode}
\usepackage{tikz}
\usepackage{hyperref}
\usepackage{url}
\usepackage[inline]{enumitem} % Ermöglicht ändern der enum Item Zahlen
\usepackage[headsepline]{scrlayer-scrpage}
\pagestyle{scrheadings}
\usetikzlibrary{automata,positioning}
\usepackage[binary-units]{siunitx}
\usepackage{tabularx}
% LaTeX Template für Abgaben an der Universität Stuttgart
% Autor: Sandro Speth
% Bei Fragen: Sandro.Speth@studi.informatik.uni-stuttgart.de
%-----------------------------------------------------------
% Modul beinhaltet Befehl fuer Aufgabennummerierung,
% sowie die Header Informationen.

% Überschreibt enumerate Befehl, sodass 1. Ebene Items mit
\renewcommand{\theenumi}{\arabic{enumi}.}
% (a), (b), etc. nummeriert werden.
\renewcommand{\labelenumi}{\text{\theenumi}}

% Counter für das Blatt und die Aufgabennummer.
% Ersetze die Nummer des Übungsblattes und die Nummer der Aufgabe
% den Anforderungen entsprechend.
% Gesetz werden die counter in der hauptdatei, damit siese hier nicht jedes mal verändert werden muss
% Beachte:
% \setcounter{countername}{number}: Legt den Wert des Counters fest
% \stepcounter{countername}: Erhöht den Wert des Counters um 1.
\newcounter{sheetnr}
\newcounter{exnum}

% Befehl für die Aufgabentitel
\newcommand{\exercise}[1]{\section*{Aufgabe \theexnum\stepcounter{exnum}: #1}} % Befehl für Aufgabentitel

% Formatierung der Kopfzeile
% \ohead: Setzt rechten Teil der Kopfzeile mit
% Namen und Matrikelnummern aller Bearbeiter
\ohead{Alexander Kharitonov (3390885)\\
       Thomas Oßwald (3307917)\\
       Angela Kächele (3260892)}

% \chead{} kann mittleren Kopfzeilen Teil sezten
% \ihead: Setzt linken Teil der Kopfzeile mit
% Modulnamen, Semester und Übungsblattnummer
\ihead{Rechnernetze I\\
Sommersemester 2021\\
Übungsblatt \thesheetnr}

% Counter für das Blatt und die Aufgabennummer.
% Ersetze die Nummer des Übungsblattes und die Nummer der Aufgabe
% den Anforderungen entsprechend.
% Definiert werden die Counter in FormatAndHeader.tex
% Beachte:
% \setcounter{countername}{number}: Legt den Wert des Counters fest
% \stepcounter{countername}: Erhöht den Wert des Counters um 1.
\setcounter{sheetnr}{1} % Nummer des Übungsblattes
\setcounter{exnum}{1} % Nummer der Aufgabe

% Beginn des eigentlichen Dokuments
\begin{document}

% Nutze den \exercise{Aufgabenname} Befehl, um eine neue Aufgabe zu beginnen.
% Möchtest du eine Aufgabe in der Nummerierung überspringen, schreibe vor der Aufgabe: \stepcounter{exnum}
% Möchtest du die Nummer einer Aufgabe auf eine beliebige Zahl x setzen, schreibe vor der Aufgabe: \setcounter{exnum}{x}
\setcounter{exnum}{1}
\exercise{Schichtenmodell}

\begin{enumerate}[label=(\alph*)]
    \item   Transport Control Protocol (TCP): Transport Layer \\
            Internet Protocol Version 4 (IPV4): Network Layer \\
            Ethernet: Data Link \\
            Frame: Data Link 
    \item   An dem aufgezeichneten Kommunkationsvorgang sind zwei Teilnehmer\\
            beteiligt (IP 213.234.110.251 und IP 213.234.110.145). 
    \item   Im Kommunkationsvorgang werden drei Nutznachrichten gesendet. \\
            Nachricht 1: Hallo? \\
            Nachricht 2: Was passiert? \\
            Nachricht 3: Rechnernetze!
\end{enumerate}
 
\setcounter{exnum}{2}
\exercise{Digitale Übertragung}
\newcommand{\encodinggraph}[3]{
    \draw[help lines] (0,0) grid (8,2);
    \draw[very thin] (0,0) rectangle (8,2);
    \node[] at (-0.5, 0) {\small #1};
    \node[] at (-0.5, 1) {\small #2};
    \node[] at (-0.5, 2) {\small #3};
    \node[rotate=90] at (-1.25, 1) {Amplitude};
    \foreach \i/\x in {0/1,1/1,2/1,3/1,4/1,5/0,6/1,7/0}{
            \node[] at (0.5 + \i, -0.25) {\small \x};
        }
}
\newcommand{\up}{++(0,2)}
\newcommand{\down}{++(0,-2)}
\newcommand{\step}{++(1,0)}
\newcommand{\hstep}{++(0.5,0)}

\begin{enumerate}[label=(\alph*)]
    \item Non-Return-To-Zero (NRZ) Codierung\\
          \begin{tikzpicture}
              \encodinggraph{}{0.5}{1}
              \draw[very thick, blue] (0,2) -- ++(5,0) -- \down -- \step -- \up -- \step -- \down -- \step;
          \end{tikzpicture}
    \item NRZ-Invert Codierung\\
          \begin{tikzpicture}
              \encodinggraph{}{0.5}{1}
              \draw[very thick, blue] (0,0) -- ++(0.5,0) -- \up -- \step -- \down -- \step -- \up -- \step -- \down -- \step -- \up -- \step -- \step -- \down -- ++(1.5,0);
          \end{tikzpicture}
    \item Manchester-Codierung\\
          \begin{tikzpicture}
              \encodinggraph{}{0.5}{1}
              \draw[very thick, blue] (0,2) -- \hstep -- \down -- \hstep -- \up -- \hstep -- \down -- \hstep -- \up -- \hstep -- \down -- \hstep -- \up -- \hstep -- \down -- \hstep -- \up -- \hstep -- \down -- \step -- \up -- \step -- \down -- \step -- \up -- \hstep;
          \end{tikzpicture}
    \item MLT3-Codierung\\
          \renewcommand{\up}{++(0,1)}
          \renewcommand{\down}{++(0,-1)}
          \begin{tikzpicture}
              \encodinggraph{-1}{0}{1}
              \draw[very thick, blue] (0,1) -- \hstep -- \up -- \step -- \down -- \step -- \down -- \step -- \up -- \step -- \up -- \step -- \step -- \down -- \step -- \hstep;
          \end{tikzpicture}
\end{enumerate}


\setcounter{exnum}{3}
\exercise{Theoreme von Fourier, etc.}
    \begin{enumerate}[label=(\alph*)]
        \item   Für die erste Übertragungsleitung kanne eine höhere Übertragungsarte erwartet werden. Aufgrund der höheren Anzahl an harmonischen
                Frequenzen in Übertragunsgleitung 1 (12 Harmonische) im Gegensatz zur Übertragungsleitung 2 (10 Harmonische) ergibt sich eine größere Bandbreite,
                welche für eine größere Übertragungsrate sorgt.
        \item   Mit einem Rauschabstand S/R = 1023 und dem zu verwendenenden Theorem von Shannon für einen verrauschten Kanal ergibt sich 
                für Übertragungsleitung 2: 
            \begin{align*}
                \text{mit } B &= 10 \cdot \frac{1}{0.05 s} = 200 \text{ Hz} \\
                \text{maximale Bitrate } &\leq B \cdot log_{2} (1+ S/R) \\
                &= 200 \cdot log_{2}(1024) \\
                &= 200 \cdot 10 \\
                &= 2000 \text{ bit/s}
            \end{align*}
        \item Mit 8 diskreten Stufen (V=8) und dem zu verwendeten Theorem von Nyquist für einen rauschfreien Kanal ergibt sich für Übertragungsleitung 1: 
            \begin{align*}
                \text{mit } B &= 12 \cdot \frac{1}{0.05 s} = 240 \text{ Hz}\\
                \text{maximale Bitrate } &\leq 2 \cdot B \cdot log_{2} V \\
                &= 2 \cdot 240 \cdot log_{2}(8) \\
                &= 480 \cdot 3 \\
                &= 1440 \text{ bit/s}\\
            \end{align*}
    \end{enumerate}
% Ende des Dokuments
\end{document}