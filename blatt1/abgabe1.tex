% LaTeX Template für Abgaben an der Universität Stuttgart
% Autor: Sandro Speth
% Bei Fragen: Sandro.Speth@studi.informatik.uni-stuttgart.de
%-----------------------------------------------------------
% Hauptmodul des Templates: Hier können andere Dateien eingebunden werden
% oder Inhalte direkt rein geschrieben werden.
% Kompiliere dieses Modul um eine PDF zu erzeugen.

% Dokumentenart. Ersetze 12pt, falls die Schriftgröße anzupassen ist.
\documentclass[12pt]{scrartcl}
% Einbinden der Pakete, des Headers und der Formatierung.
% Mit den \include und \input Befehlen können Dateien eingebunden werden:
% \include: Fügt einen Seitenumbruch nach dem Text ein
% \input: Fügt KEINEN Seitenumbruch nach dem Text ein
\input{../styles/Packages.tex}
\input{../styles/FormatAndHeader.tex}

% Counter für das Blatt und die Aufgabennummer.
% Ersetze die Nummer des Übungsblattes und die Nummer der Aufgabe
% den Anforderungen entsprechend.
% Definiert werden die Counter in FormatAndHeader.tex
% Beachte:
% \setcounter{countername}{number}: Legt den Wert des Counters fest
% \stepcounter{countername}: Erhöht den Wert des Counters um 1.
\setcounter{sheetnr}{1} % Nummer des Übungsblattes
\setcounter{exnum}{1} % Nummer der Aufgabe

% Beginn des eigentlichen Dokuments
\begin{document}

% Nutze den \exercise{Aufgabenname} Befehl, um eine neue Aufgabe zu beginnen.
% Möchtest du eine Aufgabe in der Nummerierung überspringen, schreibe vor der Aufgabe: \stepcounter{exnum}
% Möchtest du die Nummer einer Aufgabe auf eine beliebige Zahl x setzen, schreibe vor der Aufgabe: \setcounter{exnum}{x}

\setcounter{exnum}{2}
\exercise{Digitale Übertragung}
\newcommand{\encodinggraph}[3]{
    \draw[help lines] (0,0) grid (8,2);
    \draw[very thin] (0,0) rectangle (8,2);
    \node[] at (-0.5, 0) {\small #1};
    \node[] at (-0.5, 1) {\small #2};
    \node[] at (-0.5, 2) {\small #3};
    \node[rotate=90] at (-1.25, 1) {Amplitude};
    \foreach \i/\x in {0/1,1/1,2/1,3/1,4/1,5/0,6/1,7/0}{
            \node[] at (0.5 + \i, -0.25) {\small \x};
        }
}
\newcommand{\up}{++(0,2)}
\newcommand{\down}{++(0,-2)}
\newcommand{\step}{++(1,0)}
\newcommand{\hstep}{++(0.5,0)}

\begin{enumerate}[label=(\alph*)]
    \item Non-Return-To-Zero (NRZ) Codierung\\
          \begin{tikzpicture}
              \encodinggraph{}{0.5}{1}
              \draw[very thick, blue] (0,2) -- ++(5,0) -- \down -- \step -- \up -- \step -- \down -- \step;
          \end{tikzpicture}
    \item NRZ-Invert Codierung\\
          \begin{tikzpicture}
              \encodinggraph{}{0.5}{1}
              \draw[very thick, blue] (0,0) -- ++(0.5,0) -- \up -- \step -- \down -- \step -- \up -- \step -- \down -- \step -- \up -- \step -- \step -- \down -- ++(1.5,0);
          \end{tikzpicture}
    \item Manchester-Codierung\\
          \begin{tikzpicture}
              \encodinggraph{}{0.5}{1}
              \draw[very thick, blue] (0,2) -- \hstep -- \down -- \hstep -- \up -- \hstep -- \down -- \hstep -- \up -- \hstep -- \down -- \hstep -- \up -- \hstep -- \down -- \hstep -- \up -- \hstep -- \down -- \step -- \up -- \step -- \down -- \step -- \up -- \hstep;
          \end{tikzpicture}
    \item MLT3-Codierung\\
          \renewcommand{\up}{++(0,1)}
          \renewcommand{\down}{++(0,-1)}
          \begin{tikzpicture}
              \encodinggraph{-1}{0}{1}
              \draw[very thick, blue] (0,1) -- \hstep -- \up -- \step -- \down -- \step -- \down -- \step -- \up -- \step -- \up -- \step -- \step -- \down -- \step -- \hstep;
          \end{tikzpicture}
\end{enumerate}

% Ende des Dokuments
\end{document}